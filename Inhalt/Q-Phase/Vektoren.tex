\section{Vektoren}
	\subsection{Skalarprodukt}
		Mithilfe des Skalarprodukts kann man überprüfen ob zwei Vektoren orthogonal (d.h. in einem 90° Winkel) zueinander sind.	Die allgemeine Formel lautet:
		$$
		\vec{AB} * \vec{AC} = 
		\begin{pmatrix} 
			x_{1} \\
			x_{2} \\
			x_{3} 
		\end{pmatrix}
		* 
		\begin{pmatrix} 
			y_{1} \\
			y_{2} \\
			y_{3}
		\end{pmatrix} = x_{1} \cdot y_{1}+x_{2} \cdot y_{2}+x_{3} \cdot y_{3}
		$$
		Wenn das Skalarprodukt:
			\begin{itemize}
				\item $=0 \text{ ist} \rightarrow \text{Die Vektoren liegen } \textbf{orthogonal } \text{zueinander/90°}$
				\item $\neq 0 \text{ist} \rightarrow \text{Die Vektoren liegen } \textbf{nicht } \text{orthogonal zueinander}$
			\end{itemize}
	\subsection{Lage zweier Geraden zueinander bestimmen}
	\subsection{Mittelpunkt einer Geraden bestimmen}
		$$
\vec{m} = \frac{1}{2}\cdot(\vec{b}+\vec{c})=\frac{1}{2}\cdot
\begin{pmatrix}
\begin{pmatrix} 
x_{1}+y_{1} \\
x_{2}+y_{2} \\
x_{3}+y_{3}
\end{pmatrix}
\end{pmatrix}
$$
	\subsection{Längenformel eines Vektors}
		$$
\sqrt{a^{2}+b^{2}+c^{2}}
$$
		\textbf{Beispiel:}
		$$
\sqrt{2^{2}+2^{2}+(-1)^{2}}=
\sqrt{4+4+1}=
\sqrt{9}=3
$$
	\subsection{Punktprobe}
		$$
\begin{pmatrix} 
a_{1} \\
a_{2} \\
a_{3}
\end{pmatrix}+k\cdot
\begin{pmatrix} 
b_{1} \\
b_{2} \\
b_{3}
\end{pmatrix}=
\begin{pmatrix} 
c_{1} \\
c_{2} \\
c_{3}
\end{pmatrix} \rightarrow
\begin{vmatrix}
a_{1}+b_{1}\cdot k=c_{1} \\
a_{2}+b_{1}\cdot k=c_{2} \\
a_{3}+b_{1}\cdot k=c_{3}
\end{vmatrix} \rightarrow
\begin{vmatrix}
k=x \\
k=y \\
k=z
\end{vmatrix}
$$
		\textbf{Beispiel:}
			$$
\begin{pmatrix} 
-2 \\
3 \\
1
\end{pmatrix}+k\cdot
\begin{pmatrix} 
2 \\
-5 \\
7
\end{pmatrix}=
\begin{pmatrix} 
-12 \\
23 \\
-34
\end{pmatrix} \rightarrow
\begin{vmatrix}
-2+2 \cdot k=-12\\
3-5 \cdot k=23\\
1+7 \cdot k=-34
\end{vmatrix} \rightarrow
\begin{vmatrix}
k=-5 \\
k=-4 \\
k=-5 \\
\end{vmatrix}
$$