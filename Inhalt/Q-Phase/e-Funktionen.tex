\section{e-Funktionen}

Das gegenteil der $e$-Funktion ist die ln-Funktion

\subsection{Produktregel}
Mithilfe der Produktregel kann man $e$-Funktionen händisch ableiten. Dazu teilt man die Formel an den Malpunkten, sodass es immer nur zwei Teile gibt. Diese zwei Teile leitet man nun unabhängig voneinander ab und multipliziert die Ableitung von Teil 1 mit der nicht-Ableitung von Teil zwei welchen mit der Ableitung von Teil 2 und der nicht-Ableitung von Teil 1 addiert wird.
\\
\\
\textbf{Allgemein:} $f^{\prime}(x) = A \cdot B = A^{\prime} \cdot B  + B^{\prime} \cdot A$
\\
\\
\textbf{Beispiel:}
\\
$
f(x) = x^{2} \cdot (x-1)
\\
\text{Teil 1: }x^{2} \rightarrow 2x (Ableitung)
\\
\text{Teil 2: }(x-1) \rightarrow 1 (Ableitung)
\\
f^{\prime}(x) = 2x \cdot (x-1) + 1 \cdot x^{2} \\
= 2x^{2}-2x + x^{2} \\
= 3x^{2}-2x
$