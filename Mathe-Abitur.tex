\documentclass[a4paper,12pt]{article}
\usepackage{amsmath}
\usepackage{amsfonts}
\usepackage{amssymb}
\pagestyle{headings}
\title{Wiederholung für das Abitur im Fach Mathematik}
\date{}
\begin{document}
\maketitle
\tableofcontents
\addcontentsline{toc}{section}{Kurvendiskussion}
\section*{Kurvendiskussion}
	\addcontentsline{toc}{subsection}{Übersicht}
	\subsection*{Übersicht}
		\begin{enumerate}
			\item Definitionsbereich:
			\item Symetrie:
				\begin{enumerate}
					\item Achsensymetrie: $f(x)=f(-x)$
					\item Punktsymetrie: $f(-x) = -f(x)$
				\end{enumerate}
			\item Achsenschnittpunkt:
				\begin{enumerate}
					\item y-Achse: $f(0)$
					\item x-Achse/Nullstellen: $f(x)=0$
				\end{enumerate}
			\item Extrempunkte:
				\begin{enumerate}
					\item Notwendige Bedingung: $f^{\prime}(x)=0$
					\item Hinreichende Bedingung: $f^{\prime}(x)=0 \ \& f^{\prime\prime}(x) \neq0$
					\item Hochpunkt: $f^{\prime\prime}(x) < 0$
					\item Tiefpunkt: $f^{\prime\prime}(x)>0$
				\end{enumerate}
			\item Wendepunkte:
				\begin{enumerate}
					\item Notwendige Bedingung: $f^{\prime\prime}(x)=0$
					\item Hinreichende Bedingung: $f^{\prime\prime}(x)=0 \ \& f^{\prime\prime\prime}(x) \neq0$
					\item Links-Rechts-Wendepunkt: $f^{\prime\prime\prime}(x) < 0$
					\item Rechts-Links-Wendepunkt: $f^{\prime\prime\prime}(x)>0$
				\end{enumerate}
			\item Sattelpunkt:
				\begin{enumerate}
					\item Notwendige Bedingung: $f^{\prime}(x)=0$
					\item Hinreichende Bedingung: $f^{\prime\prime}(x)=0$
				\end{enumerate}
		\end{enumerate}
	\addcontentsline{toc}{subsection}{Nullstellen}
	\subsection*{Nullstellen}
		\addcontentsline{toc}{subsubsection}{PQ-Formel}
		\subsubsection*{PQ-Formel}
			$$
x_{1, 2} = - \frac{P}{2} \pm \sqrt{\frac{p^{2}}{4}-4}
			$$
		\addcontentsline{toc}{subsubsection}{Quadratische Ergänzung}
		\subsubsection*{Quadratische Ergänzung}
			binomische Formeln:
				\begin{enumerate}
					\item $(a+b)^{2} = 2a^{2}+2ab+b^{2}$
					\item $(a-b)^{2} = 2a^{2}-2ab+b^{2}$
					\item $(a+b)^{2}*(a-b)^{2}  = a^{2}-b^{2}$
				\end{enumerate}
		\addcontentsline{toc}{subsubsection}{Wendepunkte}		
		\subsubsection*{Wendepunkte}
			\begin{enumerate}
				\item Notwendige Bedingung: $f^{\prime\prime}(x)=0$
				\item Hinreichende Bedingung: $f^{\prime\prime\prime}(x)=+\neq0 \rightarrow rechts-links-Wendepunkt´$
				\item Einsetzen in f(x): $f(x)=a$
			\end{enumerate}

\addcontentsline{toc}{section}{Vektoren}
\section*{Vektoren}
	\addcontentsline{toc}{subsection}{Skalarprodukt}
	\subsection*{Skalarprodukt}
		$$
		\vec{AB} \cdot \vec{AC} = 
		\begin{pmatrix} 
			x_{1} \\
			x_{2} \\
			x_{3} 
		\end{pmatrix}
		\cdot 
		\begin{pmatrix} 
			y_{1} \\
			y_{2} \\
			y_{3}
		\end{pmatrix} = x_{1} \cdot y_{1}+x_{2} \cdot y_{2}+x_{3} \cdot y_{3}
		$$
		Wenn das Skalarprodukt:
			\begin{itemize}
				\item $=0 \text{ ist} \rightarrow \text{Die Vektoren liegen } \textbf{orthogonal } \text{zueinander/90°}$
				\item $\neq 0 \text{ist} \rightarrow \text{Die Vektoren liegen } \textbf{nicht } \text{orthogonal zueinander}$
			\end{itemize}
	\addcontentsline{toc}{subsection}{Lage zweier Geraden zueinander bestimmen}
	\subsection*{Lage zweier Geraden zueinander bestimmen}
	\addcontentsline{toc}{subsection}{Mittelpunkt einer Geraden bestimmen}
	\subsection*{Mittelpunkt einer Geraden bestimmen}
		$$
\vec{m} = \frac{1}{2}*(\vec{b}+\vec{c})=\frac{1}{2}*
\begin{pmatrix}
\begin{pmatrix} 
x_{1}+y_{1} \\
x_{2}+y_{2} \\
x_{3}+y_{3}
\end{pmatrix}
\end{pmatrix}
$$
	\addcontentsline{toc}{subsection}{Längenformel eines Vektors}
	\subsection*{Längenformel eines Vektors}
		$$
\sqrt{a^{2}+b^{2}+c^{2}}
$$
		\textbf{Beispiel:}
		$$
\sqrt{2^{2}+2^{2}+(-1)^{2}}=
\sqrt{4+4+1}=
\sqrt{9}=3
$$
	\addcontentsline{toc}{subsection}{Punktprobe}
	\subsection*{Punktprobe}
		$$
\begin{pmatrix} 
a_{1} \\
a_{2} \\
a_{3}
\end{pmatrix}+k*
\begin{pmatrix} 
b_{1} \\
b_{2} \\
b_{3}
\end{pmatrix}=
\begin{pmatrix} 
c_{1} \\
c_{2} \\
c_{3}
\end{pmatrix} \rightarrow
\begin{vmatrix}
a_{1}+b_{1}*k=c_{1} \\
a_{2}+b_{1}*k=c_{2} \\
a_{3}+b_{1}*k=c_{3}
\end{vmatrix} \rightarrow
\begin{vmatrix}
k=x \\
k=y \\
k=z
\end{vmatrix}
$$
		\textbf{Beispiel:}
			$$
\begin{pmatrix} 
-2 \\
3 \\
1
\end{pmatrix}+k*
\begin{pmatrix} 
2 \\
-5 \\
7
\end{pmatrix}=
\begin{pmatrix} 
-12 \\
23 \\
-34
\end{pmatrix} \rightarrow
\begin{vmatrix}
-2+2*k=-12\\
3-5*k=23\\
1+7*k=-34
\end{vmatrix} \rightarrow
\begin{vmatrix}
k=-5 \\
k=-4 \\
k=-5 \\
\end{vmatrix}
$$
\addcontentsline{toc}{section}{Stochastik}
\section*{Stochastik}
	\addcontentsline{toc}{subsection}{Empirische Standardabweichung}
	\subsection*{Empirische Standardabweichung}
		$$ \overline{s} = \sqrt{p_{1} \cdot (x_{1}-\overline{x})^{2}+p_{2} \cdot (x_{2}-\overline{x})^{2}+p_{3} \cdot (x-\overline{x})^{3}+...} $$
	\addcontentsline{toc}{subsection}{Erwartungswert}	
	\subsection*{Erwartungswert}
		$$ E(x) = 1 \cdot P(X = 1) + 2 \cdot P(X = 2) + 3 \cdot P(X = 3) + ... $$
	\addcontentsline{toc}{subsection}{Binomialkoeffizient}	
	\subsection*{Binomialkoeffizient}
$$
\binom{n}{k} =  \frac{n!}{k!\,(n-k)!}
$$

$$
\text{binomPDF: } P(X=Y) = \binom{n}{k} \cdot p^{k} \cdot (1-p)^{n-k}
$$

$$
\text{binomCDF: } P(X\leq Y) = p(x=y) + p(x=y-1) + ... + p(x=y-y)
$$
	\addcontentsline{toc}{subsection}{Vier-Felder-Tafel}	
	\subsection*{Vier-Felder-Tafel}
	\begin{center}
	\begin{tabular}{|c|c|c|c|}
	\hline 
	 & $B$ & $\overline{B}$ &  \\ 
	\hline
	$A$ & Wahrscheinlichkeit $AB$ & Wahrscheinlichkeit $A\overline{B}$ & Wahrscheinlichkeit $A$ \\ 
	\hline
	$\overline{A}$ & Wahrscheinlichkeit $\overline{A}B$ & Wahrscheinlichkeit $\overline{AB}$ & Wahrscheinlichkeit $\overline{A}$ \\ 
	\hline
	 & Wahrscheinlichkeit $B$ & Wahrscheinlichkeit $\overline{B}$ & $1$ \\ 
	\hline 
	\end{tabular} 
	\end{center}
	\textbf{Beispiel:} \\
	\\
	\begin{tabular}{|c|c|c|c|}
	\hline 
	 & $B$ & $\overline{B}$ &  \\ 
	\hline
	$A$ & $0.21$ & $0.49$ & $0.7$ \\ 
	\hline
	$\overline{A}$ & $0.06$ & $0.24$ & $0.3$ \\ 
	\hline
	 & $0.27$ & $0.73$ & $1$ \\ 
	\hline 
	\end{tabular}
	\addcontentsline{toc}{subsection}{Sigma-Regeln}
	\subsection*{Sigma-Regeln}
	\addcontentsline{toc}{subsection}{Intervalle abschätzen für sigma}
	\subsubsection*{Intervalle abschätzen für sigma}
	$$
	90% \rightarrow 1.64 \cdot \sigma \ 95% \rightarrow 1.96 \cdot \sigma \ 99% \rightarrow 2.58 \cdot \sigma
	$$
	\addcontentsline{toc}{subsection}{Erwartungswert}
	\subsection*{Erwartungswert}
	$$ \mu = n \cdot p $$
	\addcontentsline{toc}{subsection}{Standardabweichung}
	\subsection*{Standardabweichung}
	$$ \sigma = \sqrt{n \cdot p \cdot (1-p)} $$
	\addcontentsline{toc}{subsection}{Sigma-Regeln anwenden}
	\subsection*{Sigma-Regeln andwenden}
		\begin{enumerate}
			\item Gegeben: n, p
			\item $ \mu = n \cdot p$
			\item $\sigma = \sqrt{n \cdot p \cdot (1-p)}$ $\rightarrow$ Wenn $\sigma$ > 3 ist:
			\begin{enumerate}
				\item $1.64 \cdot \sigma = d$
				\item $\mu -d \leq X \leq \mu + d$
				\item $P(\mu + d (aufrunden) < X <\mu + d (abrunden))$
				\item $P(\mu + d (abrunden) < X <\mu + d (aufrunden))$
				\item Hinweis: Dies kann mit dem binomCDF befehl des CAS berechnet berechnet werden.
				\item Das Ergebnis welches am nächsten über 0.9 liegt ist das bessere Ergebnis
			\end{enumerate}
		\end{enumerate}
	\textbf{Beispiel:}
	\begin{enumerate}
			\item Gegeben: $n = 920$,  $p = 58$
			\item $ \mu = 920 \cdot 0.58 = 533.6$
			\item $\sigma = \sqrt{920 \cdot 0.58 \cdot (0.42)} = 14.9703$ $\rightarrow$ Wenn $\sigma$ > 3 ist:
			\begin{enumerate}
				\item $1.64 \cdot 14.9703 = 24.5513$
				\item $533.6 - 24.5513 \leq X \leq 533.5 + 24.5513 \rightarrow 509.0487 \leq X \leq 558.1513$
				\item $P(510 < X < 558) = 0.8982$
				\item $P(509 < X <559) = 0.9114$
				\item $\rightarrow$ Das richtige Ergebnis ist $P(509 \leq X \leq 559)$
			\end{enumerate}
		\end{enumerate}
	\addcontentsline{toc}{section}{Bearbeitung einer Textaufgabe in der Klausur}
	\section*{Bearbeitung einer Textaufgabe in der Klausur}
	\begin{enumerate}
		\item $f(x) \ f^{\prime}(x) \ f^{\prime\prime}(x) \ f^{\prime\prime\prime}(x)$ hinschreiben und im CAS definieren
		\item Worfür steht x bzw. t, Wofür steht $f(x)$ bzw. $A(t)$ $\rightarrow$ Was bedeutet $f^{\prime}(x)$ bzw. $A^{\prime}(x)$
		\item Teilaufgaben genau lesen: Ist x \textbf{gegeben} oder \textbf{gesucht}? Ist f(x) \textbf{gegeben} oder \textbf{gesucht}?	
	\end{enumerate}	
\end{document}