\documentclass[a4paper,12pt]{article}
\usepackage{amsmath}
\begin{document}
\pagestyle{headings}
\section*{Kurvendiskussion}
	\subsection*{Übersicht}
		\begin{enumerate}
			\item Definitionsbereich:
			\item Symetrie:
				\begin{enumerate}
					\item Achsensymetrie: $f(x)=f(-x)$
					\item Punktsymetrie: $f(-x) = -f(x)$
				\end{enumerate}
			\item Achsenschnittpunkt:
				\begin{enumerate}
					\item y-Achse: $f(0)$
					\item x-Achse/Nullstellen: $f(x)=0$
				\end{enumerate}
			\item Extrempunkte:
				\begin{enumerate}
					\item Notwendige Bedingung: $f^{\prime}(x)=0$
					\item Hinreichende Bedingung: $f^{\prime}(x)=0 \ \& f^{\prime\prime}(x) \neq0$
					\item Hochpunkt: $f^{\prime\prime}(x) < 0$
					\item Tiefpunkt: $f^{\prime\prime}(x)>0$
				\end{enumerate}
			\item Wendepunkte:
				\begin{enumerate}
					\item Notwendige Bedingung: $f^{\prime\prime}(x)=0$
					\item Hinreichende Bedingung: $f^{\prime\prime}(x)=0 \ \& f^{\prime\prime\prime}(x) \neq0$
					\item Links-Rechts-Wendepunkt: $f^{\prime\prime\prime}(x) < 0$
					\item Rechts-Links-Wendepunkt: $f^{\prime\prime\prime}(x)>0$
				\end{enumerate}
			\item Sattelpunkt:
				\begin{enumerate}
					\item Notwendige Bedingung: $f^{\prime}(x)=0$
					\item Hinreichende Bedingung: $f^{\prime\prime}(x)=0$
				\end{enumerate}
		\end{enumerate}
	\subsection*{Nullstellen}
		\subsubsection*{PQ-Formel}
			$$
x_{1, 2} = - \frac{P}{2} \pm \sqrt{\frac{p^{2}}{4}-4}
			$$
		\subsubsection*{Quadratische Ergänzung}
			binomische Formeln:
				\begin{enumerate}
					\item $(a+b)^{2} = 2a^{2}+2ab+b^{2}$
					\item $(a-b)^{2} = 2a^{2}-2ab+b^{2}$
					\item $(a+b)^{2}*(a-b)^{2}  = a^{2}-b^{2}$
				\end{enumerate}
		\subsubsection*{Wendepunkte}
			\begin{enumerate}
				\item Notwendige Bedingung: $f^{\prime\prime}(x)=0$
				\item Hinreichende Bedingung: $f^{\prime\prime\prime}(x)=+\neq0 \rightarrow rechts-links-Wendepunkt´$
				\item Einsetzen in f(x): $f(x)=a$
			\end{enumerate}
\section*{Vektoren}
	\subsection*{Skalarprodukt}
		$$
		\vec{AB} \cdot \vec{AC} = 
		\begin{pmatrix} 
			x_{1} \\
			x_{2} \\
			x_{3} 
		\end{pmatrix}
		\cdot 
		\begin{pmatrix} 
			y_{1} \\
			y_{2} \\
			y_{3}
		\end{pmatrix} = x_{1} \cdot y_{1}+x_{2} \cdot y_{2}+x_{3} \cdot y_{3}
		$$
		Wenn das Skalarprodukt:
			\begin{itemize}
				\item $=0 \text{ ist} \rightarrow \text{Die Vektoren liegen } \textbf{orthogonal } \text{zueinander/90°}$
				\item $\neq 0 \text{ist} \rightarrow \text{Die Vektoren liegen } \textbf{nicht } \text{orthogonal zueinander}$
			\end{itemize}
	\subsection*{Lage zweier Geraden zueinander bestimmen}
	\subsection*{Mittelpunkt einer Geraden bestimmen}
		$$
\vec{m} = \frac{1}{2}*(\vec{b}+\vec{c})=\frac{1}{2}*
\begin{pmatrix}
\begin{pmatrix} 
x_{1}+y_{1} \\
x_{2}+y_{2} \\
x_{3}+y_{3}
\end{pmatrix}
\end{pmatrix}
$$
	\subsection*{Längenformel eines Vektors}
		$$
\sqrt{a^{2}+b^{2}+c^{2}}
$$
		\textbf{Beispiel:}
		$$
\sqrt{2^{2}+2^{2}+(-1)^{2}}=
\sqrt{4+4+1}=
\sqrt{9}=3
$$
	\subsection*{Punktprobe}
		$$
\begin{pmatrix} 
a_{1} \\
a_{2} \\
a_{3}
\end{pmatrix}+k*
\begin{pmatrix} 
b_{1} \\
b_{2} \\
b_{3}
\end{pmatrix}=
\begin{pmatrix} 
c_{1} \\
c_{2} \\
c_{3}
\end{pmatrix} \rightarrow
\begin{vmatrix}
a_{1}+b_{1}*k=c_{1} \\
a_{2}+b_{1}*k=c_{2} \\
a_{3}+b_{1}*k=c_{3}
\end{vmatrix} \rightarrow
\begin{vmatrix}
k=x \\
k=y \\
k=z
\end{vmatrix}
$$
		\textbf{Beispiel:}
			$$
\begin{pmatrix} 
-2 \\
3 \\
1
\end{pmatrix}+k*
\begin{pmatrix} 
2 \\
-5 \\
7
\end{pmatrix}=
\begin{pmatrix} 
-12 \\
23 \\
-34
\end{pmatrix} \rightarrow
\begin{vmatrix}
-2+2*k=-12\\
3-5*k=23\\
1+7*k=-34
\end{vmatrix} \rightarrow
\begin{vmatrix}
k=-5 \\
k=-4 \\
k=-5
\end{vmatrix}
$$
\section*{Stochastik}
	\subsection*{Empirische Standardabweichung}
			
\end{document}